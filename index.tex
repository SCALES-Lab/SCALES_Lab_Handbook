% Options for packages loaded elsewhere
\PassOptionsToPackage{unicode}{hyperref}
\PassOptionsToPackage{hyphens}{url}
\PassOptionsToPackage{dvipsnames,svgnames,x11names}{xcolor}
%
\documentclass[
  letterpaper,
  DIV=11,
  numbers=noendperiod]{scrreprt}

\usepackage{amsmath,amssymb}
\usepackage{iftex}
\ifPDFTeX
  \usepackage[T1]{fontenc}
  \usepackage[utf8]{inputenc}
  \usepackage{textcomp} % provide euro and other symbols
\else % if luatex or xetex
  \usepackage{unicode-math}
  \defaultfontfeatures{Scale=MatchLowercase}
  \defaultfontfeatures[\rmfamily]{Ligatures=TeX,Scale=1}
\fi
\usepackage{lmodern}
\ifPDFTeX\else  
    % xetex/luatex font selection
\fi
% Use upquote if available, for straight quotes in verbatim environments
\IfFileExists{upquote.sty}{\usepackage{upquote}}{}
\IfFileExists{microtype.sty}{% use microtype if available
  \usepackage[]{microtype}
  \UseMicrotypeSet[protrusion]{basicmath} % disable protrusion for tt fonts
}{}
\makeatletter
\@ifundefined{KOMAClassName}{% if non-KOMA class
  \IfFileExists{parskip.sty}{%
    \usepackage{parskip}
  }{% else
    \setlength{\parindent}{0pt}
    \setlength{\parskip}{6pt plus 2pt minus 1pt}}
}{% if KOMA class
  \KOMAoptions{parskip=half}}
\makeatother
\usepackage{xcolor}
\setlength{\emergencystretch}{3em} % prevent overfull lines
\setcounter{secnumdepth}{5}
% Make \paragraph and \subparagraph free-standing
\makeatletter
\ifx\paragraph\undefined\else
  \let\oldparagraph\paragraph
  \renewcommand{\paragraph}{
    \@ifstar
      \xxxParagraphStar
      \xxxParagraphNoStar
  }
  \newcommand{\xxxParagraphStar}[1]{\oldparagraph*{#1}\mbox{}}
  \newcommand{\xxxParagraphNoStar}[1]{\oldparagraph{#1}\mbox{}}
\fi
\ifx\subparagraph\undefined\else
  \let\oldsubparagraph\subparagraph
  \renewcommand{\subparagraph}{
    \@ifstar
      \xxxSubParagraphStar
      \xxxSubParagraphNoStar
  }
  \newcommand{\xxxSubParagraphStar}[1]{\oldsubparagraph*{#1}\mbox{}}
  \newcommand{\xxxSubParagraphNoStar}[1]{\oldsubparagraph{#1}\mbox{}}
\fi
\makeatother

\usepackage{color}
\usepackage{fancyvrb}
\newcommand{\VerbBar}{|}
\newcommand{\VERB}{\Verb[commandchars=\\\{\}]}
\DefineVerbatimEnvironment{Highlighting}{Verbatim}{commandchars=\\\{\}}
% Add ',fontsize=\small' for more characters per line
\usepackage{framed}
\definecolor{shadecolor}{RGB}{241,243,245}
\newenvironment{Shaded}{\begin{snugshade}}{\end{snugshade}}
\newcommand{\AlertTok}[1]{\textcolor[rgb]{0.68,0.00,0.00}{#1}}
\newcommand{\AnnotationTok}[1]{\textcolor[rgb]{0.37,0.37,0.37}{#1}}
\newcommand{\AttributeTok}[1]{\textcolor[rgb]{0.40,0.45,0.13}{#1}}
\newcommand{\BaseNTok}[1]{\textcolor[rgb]{0.68,0.00,0.00}{#1}}
\newcommand{\BuiltInTok}[1]{\textcolor[rgb]{0.00,0.23,0.31}{#1}}
\newcommand{\CharTok}[1]{\textcolor[rgb]{0.13,0.47,0.30}{#1}}
\newcommand{\CommentTok}[1]{\textcolor[rgb]{0.37,0.37,0.37}{#1}}
\newcommand{\CommentVarTok}[1]{\textcolor[rgb]{0.37,0.37,0.37}{\textit{#1}}}
\newcommand{\ConstantTok}[1]{\textcolor[rgb]{0.56,0.35,0.01}{#1}}
\newcommand{\ControlFlowTok}[1]{\textcolor[rgb]{0.00,0.23,0.31}{\textbf{#1}}}
\newcommand{\DataTypeTok}[1]{\textcolor[rgb]{0.68,0.00,0.00}{#1}}
\newcommand{\DecValTok}[1]{\textcolor[rgb]{0.68,0.00,0.00}{#1}}
\newcommand{\DocumentationTok}[1]{\textcolor[rgb]{0.37,0.37,0.37}{\textit{#1}}}
\newcommand{\ErrorTok}[1]{\textcolor[rgb]{0.68,0.00,0.00}{#1}}
\newcommand{\ExtensionTok}[1]{\textcolor[rgb]{0.00,0.23,0.31}{#1}}
\newcommand{\FloatTok}[1]{\textcolor[rgb]{0.68,0.00,0.00}{#1}}
\newcommand{\FunctionTok}[1]{\textcolor[rgb]{0.28,0.35,0.67}{#1}}
\newcommand{\ImportTok}[1]{\textcolor[rgb]{0.00,0.46,0.62}{#1}}
\newcommand{\InformationTok}[1]{\textcolor[rgb]{0.37,0.37,0.37}{#1}}
\newcommand{\KeywordTok}[1]{\textcolor[rgb]{0.00,0.23,0.31}{\textbf{#1}}}
\newcommand{\NormalTok}[1]{\textcolor[rgb]{0.00,0.23,0.31}{#1}}
\newcommand{\OperatorTok}[1]{\textcolor[rgb]{0.37,0.37,0.37}{#1}}
\newcommand{\OtherTok}[1]{\textcolor[rgb]{0.00,0.23,0.31}{#1}}
\newcommand{\PreprocessorTok}[1]{\textcolor[rgb]{0.68,0.00,0.00}{#1}}
\newcommand{\RegionMarkerTok}[1]{\textcolor[rgb]{0.00,0.23,0.31}{#1}}
\newcommand{\SpecialCharTok}[1]{\textcolor[rgb]{0.37,0.37,0.37}{#1}}
\newcommand{\SpecialStringTok}[1]{\textcolor[rgb]{0.13,0.47,0.30}{#1}}
\newcommand{\StringTok}[1]{\textcolor[rgb]{0.13,0.47,0.30}{#1}}
\newcommand{\VariableTok}[1]{\textcolor[rgb]{0.07,0.07,0.07}{#1}}
\newcommand{\VerbatimStringTok}[1]{\textcolor[rgb]{0.13,0.47,0.30}{#1}}
\newcommand{\WarningTok}[1]{\textcolor[rgb]{0.37,0.37,0.37}{\textit{#1}}}

\providecommand{\tightlist}{%
  \setlength{\itemsep}{0pt}\setlength{\parskip}{0pt}}\usepackage{longtable,booktabs,array}
\usepackage{calc} % for calculating minipage widths
% Correct order of tables after \paragraph or \subparagraph
\usepackage{etoolbox}
\makeatletter
\patchcmd\longtable{\par}{\if@noskipsec\mbox{}\fi\par}{}{}
\makeatother
% Allow footnotes in longtable head/foot
\IfFileExists{footnotehyper.sty}{\usepackage{footnotehyper}}{\usepackage{footnote}}
\makesavenoteenv{longtable}
\usepackage{graphicx}
\makeatletter
\newsavebox\pandoc@box
\newcommand*\pandocbounded[1]{% scales image to fit in text height/width
  \sbox\pandoc@box{#1}%
  \Gscale@div\@tempa{\textheight}{\dimexpr\ht\pandoc@box+\dp\pandoc@box\relax}%
  \Gscale@div\@tempb{\linewidth}{\wd\pandoc@box}%
  \ifdim\@tempb\p@<\@tempa\p@\let\@tempa\@tempb\fi% select the smaller of both
  \ifdim\@tempa\p@<\p@\scalebox{\@tempa}{\usebox\pandoc@box}%
  \else\usebox{\pandoc@box}%
  \fi%
}
% Set default figure placement to htbp
\def\fps@figure{htbp}
\makeatother
% definitions for citeproc citations
\NewDocumentCommand\citeproctext{}{}
\NewDocumentCommand\citeproc{mm}{%
  \begingroup\def\citeproctext{#2}\cite{#1}\endgroup}
\makeatletter
 % allow citations to break across lines
 \let\@cite@ofmt\@firstofone
 % avoid brackets around text for \cite:
 \def\@biblabel#1{}
 \def\@cite#1#2{{#1\if@tempswa , #2\fi}}
\makeatother
\newlength{\cslhangindent}
\setlength{\cslhangindent}{1.5em}
\newlength{\csllabelwidth}
\setlength{\csllabelwidth}{3em}
\newenvironment{CSLReferences}[2] % #1 hanging-indent, #2 entry-spacing
 {\begin{list}{}{%
  \setlength{\itemindent}{0pt}
  \setlength{\leftmargin}{0pt}
  \setlength{\parsep}{0pt}
  % turn on hanging indent if param 1 is 1
  \ifodd #1
   \setlength{\leftmargin}{\cslhangindent}
   \setlength{\itemindent}{-1\cslhangindent}
  \fi
  % set entry spacing
  \setlength{\itemsep}{#2\baselineskip}}}
 {\end{list}}
\usepackage{calc}
\newcommand{\CSLBlock}[1]{\hfill\break\parbox[t]{\linewidth}{\strut\ignorespaces#1\strut}}
\newcommand{\CSLLeftMargin}[1]{\parbox[t]{\csllabelwidth}{\strut#1\strut}}
\newcommand{\CSLRightInline}[1]{\parbox[t]{\linewidth - \csllabelwidth}{\strut#1\strut}}
\newcommand{\CSLIndent}[1]{\hspace{\cslhangindent}#1}

\KOMAoption{captions}{tableheading}
\makeatletter
\@ifpackageloaded{bookmark}{}{\usepackage{bookmark}}
\makeatother
\makeatletter
\@ifpackageloaded{caption}{}{\usepackage{caption}}
\AtBeginDocument{%
\ifdefined\contentsname
  \renewcommand*\contentsname{Table of contents}
\else
  \newcommand\contentsname{Table of contents}
\fi
\ifdefined\listfigurename
  \renewcommand*\listfigurename{List of Figures}
\else
  \newcommand\listfigurename{List of Figures}
\fi
\ifdefined\listtablename
  \renewcommand*\listtablename{List of Tables}
\else
  \newcommand\listtablename{List of Tables}
\fi
\ifdefined\figurename
  \renewcommand*\figurename{Figure}
\else
  \newcommand\figurename{Figure}
\fi
\ifdefined\tablename
  \renewcommand*\tablename{Table}
\else
  \newcommand\tablename{Table}
\fi
}
\@ifpackageloaded{float}{}{\usepackage{float}}
\floatstyle{ruled}
\@ifundefined{c@chapter}{\newfloat{codelisting}{h}{lop}}{\newfloat{codelisting}{h}{lop}[chapter]}
\floatname{codelisting}{Listing}
\newcommand*\listoflistings{\listof{codelisting}{List of Listings}}
\makeatother
\makeatletter
\makeatother
\makeatletter
\@ifpackageloaded{caption}{}{\usepackage{caption}}
\@ifpackageloaded{subcaption}{}{\usepackage{subcaption}}
\makeatother

\usepackage{bookmark}

\IfFileExists{xurl.sty}{\usepackage{xurl}}{} % add URL line breaks if available
\urlstyle{same} % disable monospaced font for URLs
\hypersetup{
  pdftitle={SCALES Lab Handbook},
  pdfauthor={William Murrah},
  colorlinks=true,
  linkcolor={blue},
  filecolor={Maroon},
  citecolor={Blue},
  urlcolor={Blue},
  pdfcreator={LaTeX via pandoc}}


\title{SCALES Lab Handbook}
\author{William Murrah}
\date{2025-04-12}

\begin{document}
\maketitle

\renewcommand*\contentsname{Table of contents}
{
\hypersetup{linkcolor=}
\setcounter{tocdepth}{2}
\tableofcontents
}

\bookmarksetup{startatroot}

\chapter*{Preface}\label{preface}
\addcontentsline{toc}{chapter}{Preface}

\markboth{Preface}{Preface}

Welcome to the SCALES Lab!

\section*{Study of Cognition and Learning in Educational
Systems}\label{study-of-cognition-and-learning-in-educational-systems}
\addcontentsline{toc}{section}{Study of Cognition and Learning in
Educational Systems}

\markright{Study of Cognition and Learning in Educational Systems}

The SCALES Lab investigates how educational outcomes emerge from the
complex interplay of cognitive, social, institutional, and environmental
forces. We approach education not as a collection of isolated
interventions or outcomes, but as a dynamic system shaped by feedback
loops, structural conditions, and resource distribution.

Our research focuses on understanding these systems through rigorous,
theory-driven modeling --- integrating insights from cognitive science,
ecology, data science, and education policy. Rather than centering
normative agendas, we aim to develop tools and evidence that clarify how
learning environments function and how access to educational resources
can be broadened across diverse contexts.

\section*{Who We Are}\label{who-we-are}
\addcontentsline{toc}{section}{Who We Are}

\markright{Who We Are}

SCALES is a collaborative research space for graduate students, faculty,
and partners who are curious, rigorous, and ready to rethink how we
study education. We welcome scholars from diverse disciplines ---
especially those who care about educational equity, systems-level
change, and developing stronger methods for understanding how learning
happens.

\section*{What We Do}\label{what-we-do}
\addcontentsline{toc}{section}{What We Do}

\markright{What We Do}

• Build ecological models of learning and achievement\\
• Use AI and simulation to test educational theories\\
• Develop new tools for causal inference and data integration\\
• Mentor students in cutting-edge methods and policy-relevant research\\
• Collaborate across institutions, disciplines, and communities

\bookmarksetup{startatroot}

\chapter{Introduction}\label{introduction}

\bookmarksetup{startatroot}

\chapter{Summary}\label{summary}

In summary, this book has no content whatsoever.

\bookmarksetup{startatroot}

\chapter{Conferences}\label{conferences}

\begin{longtable}[]{@{}
  >{\raggedright\arraybackslash}p{(\linewidth - 14\tabcolsep) * \real{0.1085}}
  >{\raggedright\arraybackslash}p{(\linewidth - 14\tabcolsep) * \real{0.1550}}
  >{\raggedright\arraybackslash}p{(\linewidth - 14\tabcolsep) * \real{0.1473}}
  >{\raggedright\arraybackslash}p{(\linewidth - 14\tabcolsep) * \real{0.1628}}
  >{\raggedright\arraybackslash}p{(\linewidth - 14\tabcolsep) * \real{0.0775}}
  >{\raggedright\arraybackslash}p{(\linewidth - 14\tabcolsep) * \real{0.1008}}
  >{\raggedright\arraybackslash}p{(\linewidth - 14\tabcolsep) * \real{0.1318}}
  >{\raggedright\arraybackslash}p{(\linewidth - 14\tabcolsep) * \real{0.1163}}@{}}
\caption{SCALES Lab Recommended Professioal Organizations and
Conferences}\tabularnewline
\toprule\noalign{}
\begin{minipage}[b]{\linewidth}\raggedright
Organization
\end{minipage} & \begin{minipage}[b]{\linewidth}\raggedright
Research Alignment
\end{minipage} & \begin{minipage}[b]{\linewidth}\raggedright
Conference Timing
\end{minipage} & \begin{minipage}[b]{\linewidth}\raggedright
Submission Deadline
\end{minipage} & \begin{minipage}[b]{\linewidth}\raggedright
Location
\end{minipage} & \begin{minipage}[b]{\linewidth}\raggedright
Annual Dues
\end{minipage} & \begin{minipage}[b]{\linewidth}\raggedright
Primary Journal
\end{minipage} & \begin{minipage}[b]{\linewidth}\raggedright
Impact Factor
\end{minipage} \\
\midrule\noalign{}
\endfirsthead
\toprule\noalign{}
\begin{minipage}[b]{\linewidth}\raggedright
Organization
\end{minipage} & \begin{minipage}[b]{\linewidth}\raggedright
Research Alignment
\end{minipage} & \begin{minipage}[b]{\linewidth}\raggedright
Conference Timing
\end{minipage} & \begin{minipage}[b]{\linewidth}\raggedright
Submission Deadline
\end{minipage} & \begin{minipage}[b]{\linewidth}\raggedright
Location
\end{minipage} & \begin{minipage}[b]{\linewidth}\raggedright
Annual Dues
\end{minipage} & \begin{minipage}[b]{\linewidth}\raggedright
Primary Journal
\end{minipage} & \begin{minipage}[b]{\linewidth}\raggedright
Impact Factor
\end{minipage} \\
\midrule\noalign{}
\endhead
\bottomrule\noalign{}
\endlastfoot
\href{https://cognitivesciencesociety.org}{Cognitive Science Society
(CogSci)} & Cognitive science, computational modeling, theory
development & July 30 -- Aug 2, 2025 & Feb 3, 2025 & San Francisco, CA &
Regular: \$105; Student: \$55 &
\href{https://onlinelibrary.wiley.com/journal/15516709}{Cognitive
Science} & 2.3 \\
\href{https://www.isls.org}{International Society of the Learning
Sciences (ISLS)} & Learning sciences, educational technology,
collaborative learning & June 10--13, 2025 & Nov 27, 2024 & Helsinki,
Finland & Regular: \$100; Student: \$50 &
\href{https://www.tandfonline.com/journals/hlns20}{Journal of the
Learning Sciences} & 4.0 \\
\href{https://www.isss.org}{International Society for Systems Sciences
(ISSS)} & Systems thinking, ecological modeling, interdisciplinary
research & July 2025 & Mar--Apr 2025 & Varies annually & Regular: \$150;
Student: \$50 &
\href{https://onlinelibrary.wiley.com/journal/10991743}{Systems Research
and Behavioral Science} & 1.6 \\
\href{https://www.icabm.org}{International Congress on Agent-Based
Modeling (ICABM)} & Agent-based modeling, simulation, complex systems &
Varies & Varies & Varies & Regular: \$100; Student: \$50 &
\href{https://jasss.soc.surrey.ac.uk}{Journal of Artificial Societies
and Social Simulation} & 2.0 \\
\href{https://www.aera.net}{American Educational Research Association
(AERA)} & Education research, policy, methodology & Apr 23--27, 2025 &
Jul--Aug 2024 & Denver, CO & Regular: \$250; Student: \$75 &
\href{https://journals.sagepub.com/home/edr}{Educational Researcher} &
5.0 \\
\href{https://aefpweb.org}{Association for Education Finance and Policy
(AEFP)} & Education finance, policy analysis, quantitative methods &
March 2025 & Oct--Nov 2024 & Varies annually & Regular: \$150; Student:
\$50 & \href{https://direct.mit.edu/edfp}{Education Finance and Policy}
& 2.1 \\
\href{https://aaai.org}{Association for the Advancement of Artificial
Intelligence (AAAI)} & Artificial intelligence, machine learning, AI in
education & Feb 25 -- Mar 4, 2025 & Aug 2024 & Philadelphia, PA &
Regular: \$125; Student: \$50 &
\href{https://www.aaai.org/Magazine/magazine.php}{AI Magazine} & 1.7 \\
\href{https://educationaldatamining.org}{International Educational Data
Mining Society (IEDMS)} & Educational data mining, learning analytics,
AI applications & July 2025 & Jan--Feb 2025 & Varies annually & Regular:
\$100; Student: \$50 &
\href{https://jedm.educationaldatamining.org}{Journal of Educational
Data Mining} & 1.5 \\
\href{https://sase.org}{Society for the Advancement of Socio-Economics
(SASE)} & Socio-economic systems, interdisciplinary policy research &
June 2025 & Jan--Feb 2025 & Varies annually & Regular: \$150; Student:
\$50 & \href{https://academic.oup.com/ser}{Socio-Economic Review} &
3.2 \\
\href{https://www.sree.org}{Society for Research on Educational
Effectiveness (SREE)} & Educational effectiveness, causal inference,
policy evaluation & March 2025 & Sept--Oct 2024 & Varies annually &
Regular: \$200; Student: \$75 &
\href{https://www.tandfonline.com/journals/uree20}{Journal of Research
on Educational Effectiveness} & 2.9 \\
\href{https://imbes.org}{International Mind, Brain, and Education
Society (IMBES)} & Cognitive neuroscience, education, interdisciplinary
research & 2026 (Biennial) & TBD & Varies & Regular: \$100; Student:
\$50 & \href{https://onlinelibrary.wiley.com/journal/1751228x}{Mind,
Brain, and Education} & 2.5 \\
\end{longtable}

\bookmarksetup{startatroot}

\chapter{SCALES Project Template}\label{scales-project-template}

\href{https://scales-lab.github.io/SCALES_Lab_Handbook/}{SCALES Project
Template Github page}

Understanding the distinction between scripts and modular functions is
key to organizing a clean, scalable, and reproducible research project.
Here's a breakdown tailored to your workflow in the SCALES Lab:

\section{Scripts vs.~Modular
Functions}\label{scripts-vs.-modular-functions}

\begin{longtable}[]{@{}
  >{\raggedright\arraybackslash}p{(\linewidth - 4\tabcolsep) * \real{0.1760}}
  >{\raggedright\arraybackslash}p{(\linewidth - 4\tabcolsep) * \real{0.3760}}
  >{\raggedright\arraybackslash}p{(\linewidth - 4\tabcolsep) * \real{0.4480}}@{}}
\toprule\noalign{}
\begin{minipage}[b]{\linewidth}\raggedright
Feature
\end{minipage} & \begin{minipage}[b]{\linewidth}\raggedright
\textbf{Scripts}
\end{minipage} & \begin{minipage}[b]{\linewidth}\raggedright
\textbf{Modular Functions}
\end{minipage} \\
\midrule\noalign{}
\endhead
\bottomrule\noalign{}
\endlastfoot
\textbf{Purpose} & Perform a specific task or workflow & Define reusable
logic that can be called elsewhere \\
\textbf{Structure} & Linear and executable top-to-bottom & Encapsulated
into functions or classes \\
\textbf{Typical Location} & \texttt{scripts/} & \texttt{src/} (e.g.,
\texttt{src/r/}, \texttt{src/py/}) \\
\textbf{Example Task} & \texttt{clean\_data.R} runs the full cleaning
pipeline & \texttt{remove\_outliers()} is used inside that script \\
\textbf{Reusability} & Low --- task-specific & High --- written to be
reused in multiple scripts \\
\textbf{Execution} & Run as a whole (\texttt{python\ analyze.py}) &
Loaded or imported into other files \\
\textbf{Naming} & Verb-based (e.g., \texttt{analyze\_data.py}) &
Noun/action-based (e.g., \texttt{utils.py}, \texttt{metrics.R}) \\
\end{longtable}

\section{In Practice}\label{in-practice}

Example Script: scripts/analyze.py

\begin{Shaded}
\begin{Highlighting}[]
\ImportTok{import}\NormalTok{ pandas }\ImportTok{as}\NormalTok{ pd}
\ImportTok{from}\NormalTok{ src.py.utils }\ImportTok{import}\NormalTok{ remove\_outliers, standardize\_scores}

\NormalTok{df }\OperatorTok{=}\NormalTok{ pd.read\_csv(}\StringTok{"data/processed/student\_data.csv"}\NormalTok{)}
\NormalTok{df }\OperatorTok{=}\NormalTok{ remove\_outliers(df)}
\NormalTok{df }\OperatorTok{=}\NormalTok{ standardize\_scores(df)}
\NormalTok{df.to\_csv(}\StringTok{"data/processed/cleaned.csv"}\NormalTok{)}
\end{Highlighting}
\end{Shaded}

Example Function File: src/py/utils.py

\begin{Shaded}
\begin{Highlighting}[]
\KeywordTok{def}\NormalTok{ remove\_outliers(df, threshold}\OperatorTok{=}\DecValTok{3}\NormalTok{):}
    \ControlFlowTok{return}\NormalTok{ df[(df }\OperatorTok{\textless{}}\NormalTok{ threshold).}\BuiltInTok{all}\NormalTok{(axis}\OperatorTok{=}\DecValTok{1}\NormalTok{)]}

\KeywordTok{def}\NormalTok{ standardize\_scores(df):}
    \ControlFlowTok{return}\NormalTok{ (df }\OperatorTok{{-}}\NormalTok{ df.mean()) }\OperatorTok{/}\NormalTok{ df.std()}
\end{Highlighting}
\end{Shaded}

\section{Why This Matters for
Reproducibility}\label{why-this-matters-for-reproducibility}

\begin{verbatim}
*   Scripts make your research pipeline transparent.
*   Modular functions make your code clean, testable, and scalable.
*   This separation supports version control and collaboration — team members can modify or improve functions without altering your analytic workflow scripts.
\end{verbatim}

⸻

Would you like me to generate template function and script files in both
R and Python as part of the GitHub template repo?

\bookmarksetup{startatroot}

\chapter*{References}\label{references}
\addcontentsline{toc}{chapter}{References}

\markboth{References}{References}

\phantomsection\label{refs}
\begin{CSLReferences}{0}{1}
\end{CSLReferences}




\end{document}
